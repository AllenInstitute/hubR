\nonstopmode{}
\documentclass[a4paper]{book}
\usepackage[times,inconsolata,hyper]{Rd}
\usepackage{makeidx}
\usepackage[utf8]{inputenc} % @SET ENCODING@
% \usepackage{graphicx} % @USE GRAPHICX@
\makeindex{}
\begin{document}
\chapter*{}
\begin{center}
{\textbf{\huge Package `hubR'}}
\par\bigskip{\large \today}
\end{center}
\inputencoding{utf8}
\ifthenelse{\boolean{Rd@use@hyper}}{\hypersetup{pdftitle = {hubR: Build UCSC track hubs}}}{}\ifthenelse{\boolean{Rd@use@hyper}}{\hypersetup{pdfauthor = {Nelson Johansen}}}{}\begin{description}
\raggedright{}
\item[Title]\AsIs{Build UCSC track hubs}
\item[Version]\AsIs{0.0.0.9000}
\item[URL]\AsIs{}\url{https://github.com/}\AsIs{}
\item[Description]\AsIs{Takes a list of bigwig file names, Amazon S3 bucket info and generates a UCSC genome browser track hub along with neccessary hmac signatures.}
\item[License]\AsIs{GPL-3}
\item[Imports]\AsIs{openssl, digest, RCurl, RColorBrewer}
\item[Suggests]\AsIs{knitr, rmarkdown, testthat}
\item[Encoding]\AsIs{UTF-8}
\item[VignetteBuilder]\AsIs{knitr}
\item[Roxygen]\AsIs{list(markdown = TRUE)}
\item[RoxygenNote]\AsIs{7.1.2}
\item[NeedsCompilation]\AsIs{no}
\item[Author]\AsIs{Nelson Johansen [aut, cre]}
\item[Maintainer]\AsIs{Nelson Johansen }\email{nelson.johansen@alleninstitute.org}\AsIs{}
\end{description}
\Rdcontents{\R{} topics documented:}
\inputencoding{utf8}
\HeaderA{create.signatures}{Computes HMAC signatures}{create.signatures}
%
\begin{Description}\relax
This function takes in a vector of bigwig file names and Amazon S3 credentials to generate hmac signatures to access .bw files.
\end{Description}
%
\begin{Usage}
\begin{verbatim}
create.signatures(track.dir, secret.key, bigwigs)
\end{verbatim}
\end{Usage}
%
\begin{Arguments}
\begin{ldescription}
\item[\code{track.dir}] Amazon S3 bucket name that stores the .bw files

\item[\code{secret.key}] Amazon S3 secret access key

\item[\code{bigwigs}] Bigwig file names which exactly match those in 'track.dir'
\end{ldescription}
\end{Arguments}
\inputencoding{utf8}
\HeaderA{generate.track.hub}{Build multi-wig track hub}{generate.track.hub}
%
\begin{Description}\relax
This function loads a file as a matrix. It assumes that the first column
contains the rownames and the subsequent columns are the sample identifiers.
Any rows with duplicated row names will be dropped with the first one being
kepted.
\end{Description}
%
\begin{Usage}
\begin{verbatim}
generate.track.hub(
  hmac.encoded,
  track.dir,
  access.key,
  bigwigs,
  pseudo.names,
  long.labels,
  colors,
  species,
  region,
  type,
  cluster,
  genome,
  output.track.file,
  email
)
\end{verbatim}
\end{Usage}
%
\begin{Arguments}
\begin{ldescription}
\item[\code{hmac.encoded}] Bigwig file signatures computed with \code{create.signatures}

\item[\code{track.dir}] Amazon S3 bucket name that stores the .bw files

\item[\code{access.key}] Amazon S3 access key ID

\item[\code{bigwigs}] Bigwig file names which exactly match those in 'track.dir'

\item[\code{pseudo.names}] Short labels to give each track. If NULL the function attemps to gather from .bw file name: pseudo.names-*.bw

\item[\code{long.labels}] Detailed labels to give each track. If NULL the function attemps to gather from .bw file name: long.labels-*.bw

\item[\code{colors}] Colors (in R, G, B format) to give the tracks. If NULL colors will be auto-generated.

\item[\code{species}] Species information

\item[\code{region}] Brain region information

\item[\code{type}] Data type (ATAC, Multiome, etc.)

\item[\code{cluster}] Cluster label information

\item[\code{genome}] Genome information

\item[\code{output.track.file}] Output track hub filename. Default: trackDB.txt

\item[\code{email}] Correspondence email
\end{ldescription}
\end{Arguments}
%
\begin{Value}
A matrix of the infile
\end{Value}
\inputencoding{utf8}
\HeaderA{hubR}{Builds a track hub for the UCSC genome browser}{hubR}
%
\begin{Description}\relax
This function takes in a vector of bigwig file names along associated metadata and Amazon S3 credentials.
It assumes the user has created an S3 bucket 'track.dir' where the bigwig files will be accessible on S3.
Any changes to the bigwig file names or 'track.dir' will require the regeneration of HMAC signatures and the hub file.
\end{Description}
%
\begin{Usage}
\begin{verbatim}
hubR(
  track.dir,
  access.key,
  secret.key,
  bigwigs,
  pseudo.names = NULL,
  long.labels = NULL,
  colors = NULL,
  species,
  region,
  type,
  cluster,
  genome,
  output.track.file = "trackDB.txt",
  email = ""
)
\end{verbatim}
\end{Usage}
%
\begin{Arguments}
\begin{ldescription}
\item[\code{track.dir}] Amazon S3 bucket name that stores the .bw files

\item[\code{access.key}] Amazon S3 access key ID

\item[\code{secret.key}] Amazon S3 secret access key

\item[\code{bigwigs}] Bigwig file names which exactly match those in 'track.dir'

\item[\code{pseudo.names}] Short labels to give each track. If NULL the function attemps to gather from .bw file name: pseudo.names-*.bw

\item[\code{long.labels}] Detailed labels to give each track. If NULL the function attemps to gather from .bw file name: long.labels-*.bw

\item[\code{colors}] Colors (in R, G, B format) to give the tracks. If NULL colors will be auto-generated.

\item[\code{species}] Species information

\item[\code{region}] Brain region information

\item[\code{type}] Data type (ATAC, Multiome, etc.)

\item[\code{cluster}] Cluster label information

\item[\code{genome}] Genome information

\item[\code{output.track.file}] Output track hub filename. Default: trackDB.txt

\item[\code{email}] Correspondence email
\end{ldescription}
\end{Arguments}
\printindex{}
\end{document}
